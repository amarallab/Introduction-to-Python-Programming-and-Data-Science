
% Default to the notebook output style

    


% Inherit from the specified cell style.




    
\documentclass{article}

    
    
    \usepackage{graphicx} % Used to insert images
    \usepackage{adjustbox} % Used to constrain images to a maximum size 
    \usepackage{color} % Allow colors to be defined
    \usepackage{enumerate} % Needed for markdown enumerations to work
    \usepackage{geometry} % Used to adjust the document margins
    \usepackage{amsmath} % Equations
    \usepackage{amssymb} % Equations
    \usepackage[mathletters]{ucs} % Extended unicode (utf-8) support
    \usepackage[utf8x]{inputenc} % Allow utf-8 characters in the tex document
    \usepackage{fancyvrb} % verbatim replacement that allows latex
    \usepackage{grffile} % extends the file name processing of package graphics 
                         % to support a larger range 
    % The hyperref package gives us a pdf with properly built
    % internal navigation ('pdf bookmarks' for the table of contents,
    % internal cross-reference links, web links for URLs, etc.)
    \usepackage{hyperref}
    \usepackage{longtable} % longtable support required by pandoc >1.10
    \usepackage{booktabs}  % table support for pandoc > 1.12.2
    

    
    
    \definecolor{orange}{cmyk}{0,0.4,0.8,0.2}
    \definecolor{darkorange}{rgb}{.71,0.21,0.01}
    \definecolor{darkgreen}{rgb}{.12,.54,.11}
    \definecolor{myteal}{rgb}{.26, .44, .56}
    \definecolor{gray}{gray}{0.45}
    \definecolor{lightgray}{gray}{.95}
    \definecolor{mediumgray}{gray}{.8}
    \definecolor{inputbackground}{rgb}{.95, .95, .85}
    \definecolor{outputbackground}{rgb}{.95, .95, .95}
    \definecolor{traceback}{rgb}{1, .95, .95}
    % ansi colors
    \definecolor{red}{rgb}{.6,0,0}
    \definecolor{green}{rgb}{0,.65,0}
    \definecolor{brown}{rgb}{0.6,0.6,0}
    \definecolor{blue}{rgb}{0,.145,.698}
    \definecolor{purple}{rgb}{.698,.145,.698}
    \definecolor{cyan}{rgb}{0,.698,.698}
    \definecolor{lightgray}{gray}{0.5}
    
    % bright ansi colors
    \definecolor{darkgray}{gray}{0.25}
    \definecolor{lightred}{rgb}{1.0,0.39,0.28}
    \definecolor{lightgreen}{rgb}{0.48,0.99,0.0}
    \definecolor{lightblue}{rgb}{0.53,0.81,0.92}
    \definecolor{lightpurple}{rgb}{0.87,0.63,0.87}
    \definecolor{lightcyan}{rgb}{0.5,1.0,0.83}
    
    % commands and environments needed by pandoc snippets
    % extracted from the output of `pandoc -s`
    \DefineVerbatimEnvironment{Highlighting}{Verbatim}{commandchars=\\\{\}}
    % Add ',fontsize=\small' for more characters per line
    \newenvironment{Shaded}{}{}
    \newcommand{\KeywordTok}[1]{\textcolor[rgb]{0.00,0.44,0.13}{\textbf{{#1}}}}
    \newcommand{\DataTypeTok}[1]{\textcolor[rgb]{0.56,0.13,0.00}{{#1}}}
    \newcommand{\DecValTok}[1]{\textcolor[rgb]{0.25,0.63,0.44}{{#1}}}
    \newcommand{\BaseNTok}[1]{\textcolor[rgb]{0.25,0.63,0.44}{{#1}}}
    \newcommand{\FloatTok}[1]{\textcolor[rgb]{0.25,0.63,0.44}{{#1}}}
    \newcommand{\CharTok}[1]{\textcolor[rgb]{0.25,0.44,0.63}{{#1}}}
    \newcommand{\StringTok}[1]{\textcolor[rgb]{0.25,0.44,0.63}{{#1}}}
    \newcommand{\CommentTok}[1]{\textcolor[rgb]{0.38,0.63,0.69}{\textit{{#1}}}}
    \newcommand{\OtherTok}[1]{\textcolor[rgb]{0.00,0.44,0.13}{{#1}}}
    \newcommand{\AlertTok}[1]{\textcolor[rgb]{1.00,0.00,0.00}{\textbf{{#1}}}}
    \newcommand{\FunctionTok}[1]{\textcolor[rgb]{0.02,0.16,0.49}{{#1}}}
    \newcommand{\RegionMarkerTok}[1]{{#1}}
    \newcommand{\ErrorTok}[1]{\textcolor[rgb]{1.00,0.00,0.00}{\textbf{{#1}}}}
    \newcommand{\NormalTok}[1]{{#1}}
    
    % Define a nice break command that doesn't care if a line doesn't already
    % exist.
    \def\br{\hspace*{\fill} \\* }
    % Math Jax compatability definitions
    \def\gt{>}
    \def\lt{<}
    % Document parameters
    \title{Data-Types}
    
    
    

    % Pygments definitions
    
\makeatletter
\def\PY@reset{\let\PY@it=\relax \let\PY@bf=\relax%
    \let\PY@ul=\relax \let\PY@tc=\relax%
    \let\PY@bc=\relax \let\PY@ff=\relax}
\def\PY@tok#1{\csname PY@tok@#1\endcsname}
\def\PY@toks#1+{\ifx\relax#1\empty\else%
    \PY@tok{#1}\expandafter\PY@toks\fi}
\def\PY@do#1{\PY@bc{\PY@tc{\PY@ul{%
    \PY@it{\PY@bf{\PY@ff{#1}}}}}}}
\def\PY#1#2{\PY@reset\PY@toks#1+\relax+\PY@do{#2}}

\expandafter\def\csname PY@tok@ne\endcsname{\let\PY@bf=\textbf\def\PY@tc##1{\textcolor[rgb]{0.82,0.25,0.23}{##1}}}
\expandafter\def\csname PY@tok@mi\endcsname{\def\PY@tc##1{\textcolor[rgb]{0.40,0.40,0.40}{##1}}}
\expandafter\def\csname PY@tok@sc\endcsname{\def\PY@tc##1{\textcolor[rgb]{0.73,0.13,0.13}{##1}}}
\expandafter\def\csname PY@tok@mh\endcsname{\def\PY@tc##1{\textcolor[rgb]{0.40,0.40,0.40}{##1}}}
\expandafter\def\csname PY@tok@kc\endcsname{\let\PY@bf=\textbf\def\PY@tc##1{\textcolor[rgb]{0.00,0.50,0.00}{##1}}}
\expandafter\def\csname PY@tok@gr\endcsname{\def\PY@tc##1{\textcolor[rgb]{1.00,0.00,0.00}{##1}}}
\expandafter\def\csname PY@tok@gd\endcsname{\def\PY@tc##1{\textcolor[rgb]{0.63,0.00,0.00}{##1}}}
\expandafter\def\csname PY@tok@k\endcsname{\let\PY@bf=\textbf\def\PY@tc##1{\textcolor[rgb]{0.00,0.50,0.00}{##1}}}
\expandafter\def\csname PY@tok@kd\endcsname{\let\PY@bf=\textbf\def\PY@tc##1{\textcolor[rgb]{0.00,0.50,0.00}{##1}}}
\expandafter\def\csname PY@tok@cm\endcsname{\let\PY@it=\textit\def\PY@tc##1{\textcolor[rgb]{0.25,0.50,0.50}{##1}}}
\expandafter\def\csname PY@tok@o\endcsname{\def\PY@tc##1{\textcolor[rgb]{0.40,0.40,0.40}{##1}}}
\expandafter\def\csname PY@tok@c1\endcsname{\let\PY@it=\textit\def\PY@tc##1{\textcolor[rgb]{0.25,0.50,0.50}{##1}}}
\expandafter\def\csname PY@tok@vg\endcsname{\def\PY@tc##1{\textcolor[rgb]{0.10,0.09,0.49}{##1}}}
\expandafter\def\csname PY@tok@s1\endcsname{\def\PY@tc##1{\textcolor[rgb]{0.73,0.13,0.13}{##1}}}
\expandafter\def\csname PY@tok@si\endcsname{\let\PY@bf=\textbf\def\PY@tc##1{\textcolor[rgb]{0.73,0.40,0.53}{##1}}}
\expandafter\def\csname PY@tok@kn\endcsname{\let\PY@bf=\textbf\def\PY@tc##1{\textcolor[rgb]{0.00,0.50,0.00}{##1}}}
\expandafter\def\csname PY@tok@nf\endcsname{\def\PY@tc##1{\textcolor[rgb]{0.00,0.00,1.00}{##1}}}
\expandafter\def\csname PY@tok@cp\endcsname{\def\PY@tc##1{\textcolor[rgb]{0.74,0.48,0.00}{##1}}}
\expandafter\def\csname PY@tok@ge\endcsname{\let\PY@it=\textit}
\expandafter\def\csname PY@tok@ni\endcsname{\let\PY@bf=\textbf\def\PY@tc##1{\textcolor[rgb]{0.60,0.60,0.60}{##1}}}
\expandafter\def\csname PY@tok@gt\endcsname{\def\PY@tc##1{\textcolor[rgb]{0.00,0.27,0.87}{##1}}}
\expandafter\def\csname PY@tok@ow\endcsname{\let\PY@bf=\textbf\def\PY@tc##1{\textcolor[rgb]{0.67,0.13,1.00}{##1}}}
\expandafter\def\csname PY@tok@nd\endcsname{\def\PY@tc##1{\textcolor[rgb]{0.67,0.13,1.00}{##1}}}
\expandafter\def\csname PY@tok@gh\endcsname{\let\PY@bf=\textbf\def\PY@tc##1{\textcolor[rgb]{0.00,0.00,0.50}{##1}}}
\expandafter\def\csname PY@tok@gi\endcsname{\def\PY@tc##1{\textcolor[rgb]{0.00,0.63,0.00}{##1}}}
\expandafter\def\csname PY@tok@sb\endcsname{\def\PY@tc##1{\textcolor[rgb]{0.73,0.13,0.13}{##1}}}
\expandafter\def\csname PY@tok@cs\endcsname{\let\PY@it=\textit\def\PY@tc##1{\textcolor[rgb]{0.25,0.50,0.50}{##1}}}
\expandafter\def\csname PY@tok@c\endcsname{\let\PY@it=\textit\def\PY@tc##1{\textcolor[rgb]{0.25,0.50,0.50}{##1}}}
\expandafter\def\csname PY@tok@se\endcsname{\let\PY@bf=\textbf\def\PY@tc##1{\textcolor[rgb]{0.73,0.40,0.13}{##1}}}
\expandafter\def\csname PY@tok@w\endcsname{\def\PY@tc##1{\textcolor[rgb]{0.73,0.73,0.73}{##1}}}
\expandafter\def\csname PY@tok@nb\endcsname{\def\PY@tc##1{\textcolor[rgb]{0.00,0.50,0.00}{##1}}}
\expandafter\def\csname PY@tok@sx\endcsname{\def\PY@tc##1{\textcolor[rgb]{0.00,0.50,0.00}{##1}}}
\expandafter\def\csname PY@tok@mo\endcsname{\def\PY@tc##1{\textcolor[rgb]{0.40,0.40,0.40}{##1}}}
\expandafter\def\csname PY@tok@nl\endcsname{\def\PY@tc##1{\textcolor[rgb]{0.63,0.63,0.00}{##1}}}
\expandafter\def\csname PY@tok@vc\endcsname{\def\PY@tc##1{\textcolor[rgb]{0.10,0.09,0.49}{##1}}}
\expandafter\def\csname PY@tok@kp\endcsname{\def\PY@tc##1{\textcolor[rgb]{0.00,0.50,0.00}{##1}}}
\expandafter\def\csname PY@tok@m\endcsname{\def\PY@tc##1{\textcolor[rgb]{0.40,0.40,0.40}{##1}}}
\expandafter\def\csname PY@tok@sr\endcsname{\def\PY@tc##1{\textcolor[rgb]{0.73,0.40,0.53}{##1}}}
\expandafter\def\csname PY@tok@sh\endcsname{\def\PY@tc##1{\textcolor[rgb]{0.73,0.13,0.13}{##1}}}
\expandafter\def\csname PY@tok@na\endcsname{\def\PY@tc##1{\textcolor[rgb]{0.49,0.56,0.16}{##1}}}
\expandafter\def\csname PY@tok@nn\endcsname{\let\PY@bf=\textbf\def\PY@tc##1{\textcolor[rgb]{0.00,0.00,1.00}{##1}}}
\expandafter\def\csname PY@tok@sd\endcsname{\let\PY@it=\textit\def\PY@tc##1{\textcolor[rgb]{0.73,0.13,0.13}{##1}}}
\expandafter\def\csname PY@tok@kr\endcsname{\let\PY@bf=\textbf\def\PY@tc##1{\textcolor[rgb]{0.00,0.50,0.00}{##1}}}
\expandafter\def\csname PY@tok@err\endcsname{\def\PY@bc##1{\setlength{\fboxsep}{0pt}\fcolorbox[rgb]{1.00,0.00,0.00}{1,1,1}{\strut ##1}}}
\expandafter\def\csname PY@tok@nt\endcsname{\let\PY@bf=\textbf\def\PY@tc##1{\textcolor[rgb]{0.00,0.50,0.00}{##1}}}
\expandafter\def\csname PY@tok@nc\endcsname{\let\PY@bf=\textbf\def\PY@tc##1{\textcolor[rgb]{0.00,0.00,1.00}{##1}}}
\expandafter\def\csname PY@tok@il\endcsname{\def\PY@tc##1{\textcolor[rgb]{0.40,0.40,0.40}{##1}}}
\expandafter\def\csname PY@tok@mf\endcsname{\def\PY@tc##1{\textcolor[rgb]{0.40,0.40,0.40}{##1}}}
\expandafter\def\csname PY@tok@gu\endcsname{\let\PY@bf=\textbf\def\PY@tc##1{\textcolor[rgb]{0.50,0.00,0.50}{##1}}}
\expandafter\def\csname PY@tok@vi\endcsname{\def\PY@tc##1{\textcolor[rgb]{0.10,0.09,0.49}{##1}}}
\expandafter\def\csname PY@tok@gs\endcsname{\let\PY@bf=\textbf}
\expandafter\def\csname PY@tok@go\endcsname{\def\PY@tc##1{\textcolor[rgb]{0.53,0.53,0.53}{##1}}}
\expandafter\def\csname PY@tok@gp\endcsname{\let\PY@bf=\textbf\def\PY@tc##1{\textcolor[rgb]{0.00,0.00,0.50}{##1}}}
\expandafter\def\csname PY@tok@kt\endcsname{\def\PY@tc##1{\textcolor[rgb]{0.69,0.00,0.25}{##1}}}
\expandafter\def\csname PY@tok@s2\endcsname{\def\PY@tc##1{\textcolor[rgb]{0.73,0.13,0.13}{##1}}}
\expandafter\def\csname PY@tok@ss\endcsname{\def\PY@tc##1{\textcolor[rgb]{0.10,0.09,0.49}{##1}}}
\expandafter\def\csname PY@tok@nv\endcsname{\def\PY@tc##1{\textcolor[rgb]{0.10,0.09,0.49}{##1}}}
\expandafter\def\csname PY@tok@s\endcsname{\def\PY@tc##1{\textcolor[rgb]{0.73,0.13,0.13}{##1}}}
\expandafter\def\csname PY@tok@no\endcsname{\def\PY@tc##1{\textcolor[rgb]{0.53,0.00,0.00}{##1}}}
\expandafter\def\csname PY@tok@bp\endcsname{\def\PY@tc##1{\textcolor[rgb]{0.00,0.50,0.00}{##1}}}

\def\PYZbs{\char`\\}
\def\PYZus{\char`\_}
\def\PYZob{\char`\{}
\def\PYZcb{\char`\}}
\def\PYZca{\char`\^}
\def\PYZam{\char`\&}
\def\PYZlt{\char`\<}
\def\PYZgt{\char`\>}
\def\PYZsh{\char`\#}
\def\PYZpc{\char`\%}
\def\PYZdl{\char`\$}
\def\PYZhy{\char`\-}
\def\PYZsq{\char`\'}
\def\PYZdq{\char`\"}
\def\PYZti{\char`\~}
% for compatibility with earlier versions
\def\PYZat{@}
\def\PYZlb{[}
\def\PYZrb{]}
\makeatother


    % Exact colors from NB
    \definecolor{incolor}{rgb}{0.0, 0.0, 0.5}
    \definecolor{outcolor}{rgb}{0.545, 0.0, 0.0}



    
    % Prevent overflowing lines due to hard-to-break entities
    \sloppy 
    % Setup hyperref package
    \hypersetup{
      breaklinks=true,  % so long urls are correctly broken across lines
      colorlinks=true,
      urlcolor=blue,
      linkcolor=darkorange,
      citecolor=darkgreen,
      }
    % Slightly bigger margins than the latex defaults
    
    \geometry{verbose,tmargin=1in,bmargin=1in,lmargin=1in,rmargin=1in}
    
    

    \begin{document}
    
    
    \maketitle
    
    

    
    \section{What's a variable?}\label{whats-a-variable}

Let's start with going over one of the most basic concepts in
programming, using a \emph{variable}. You've heard this term used before
in math class (remember \(y = mx + b\)?) and our usage here is more or
less the same.

A \textbf{variable} is a \textbf{symbolic name} that is associated with
some \textbf{value}.

This is the most basic part of programming, because we want to have some
name that we can call and will return a value. We can do this simply,
just like:

    \begin{Verbatim}[commandchars=\\\{\}]
{\color{incolor}In [{\color{incolor}2}]:} \PY{n}{number} \PY{o}{=} \PY{l+m+mi}{2}
\end{Verbatim}

    Now we have created \texttt{number} as the \textbf{variable} and
assigned \texttt{2} as its \textbf{value}. As a hint, the construction
of

\emph{variable} = \emph{value}

will hold in most other programming languages too. Now at any time we
can use our variable \texttt{number} again, or just look at its value.
We can look at it's value by using the \texttt{print()} function in
Python.

    \begin{Verbatim}[commandchars=\\\{\}]
{\color{incolor}In [{\color{incolor}3}]:} \PY{k}{print}\PY{p}{(} \PY{n}{number} \PY{p}{)}
\end{Verbatim}

    \begin{Verbatim}[commandchars=\\\{\}]
2
    \end{Verbatim}

    We can also use the variable in a mathematical expression too.

    \begin{Verbatim}[commandchars=\\\{\}]
{\color{incolor}In [{\color{incolor}4}]:} \PY{n}{number} \PY{o}{*} \PY{l+m+mi}{2}
\end{Verbatim}

            \begin{Verbatim}[commandchars=\\\{\}]
{\color{outcolor}Out[{\color{outcolor}4}]:} 4
\end{Verbatim}
        
    An important thing to note though is that a variable can have its
associated value \textbf{change}. So if we reassign \texttt{number}
using our variable creation syntax above, we'll see that it's value
changes.

    \begin{Verbatim}[commandchars=\\\{\}]
{\color{incolor}In [{\color{incolor}5}]:} \PY{n}{number} \PY{o}{=} \PY{l+m+mi}{5}
        \PY{k}{print}\PY{p}{(}\PY{n}{number}\PY{p}{)}
\end{Verbatim}

    \begin{Verbatim}[commandchars=\\\{\}]
5
    \end{Verbatim}

    An important part to remember about variables is that it's up to you to
name them well. There are a number of different data types in Python,
but there's no distinction on how you have to name them. So this means
that it is up to you to give good, descriptive names to your variables.

Why is that important?

Well, the whole point of descriptive variable names is to improve
readability and understanding of code by yourself and others. We
typically think that we will remember everything that we do, but after a
month or two of doing something else it's hard to remember any one piece
of code. Good naming practices can make all the difference here.

We can even go through a quick example.

    \begin{Verbatim}[commandchars=\\\{\}]
{\color{incolor}In [{\color{incolor}6}]:} \PY{n}{number} \PY{o}{=} \PY{l+s}{\PYZsq{}}\PY{l+s}{Helen}\PY{l+s}{\PYZsq{}}
\end{Verbatim}

    Here I've gone and changed the variable \texttt{number} to stand for
someone's name. Now that isn't great because when most people see the
word \texttt{number} they expect the variable to contain some kind of
numeric value. This means that they might want to perform some math with
the variable and they would then quickly experience this.

    \begin{Verbatim}[commandchars=\\\{\}]
{\color{incolor}In [{\color{incolor}7}]:} \PY{n}{number} \PY{o}{+} \PY{l+m+mi}{2}
\end{Verbatim}

    \begin{Verbatim}[commandchars=\\\{\}]

        ---------------------------------------------------------------------------
    TypeError                                 Traceback (most recent call last)

        <ipython-input-7-e96a55dd7add> in <module>()
    ----> 1 number + 2
    

        TypeError: Can't convert 'int' object to str implicitly

    \end{Verbatim}

    An error!

I know that this is a pretty simplistic example, but just try to keep
this point in my mind as we work through these tutorials. Now, onto the
meat!

    \section{The basic data types}\label{the-basic-data-types}

Python has eight basic data types for you to use with variables. The
first four that we will cover here all allow for a variable to be a
single value. These four types are:

\begin{itemize}
\itemsep1pt\parskip0pt\parsep0pt
\item
  Integers
\item
  Floats
\item
  Strings
\item
  Booleans
\end{itemize}

The other four we will cover in the next lesson. Those types allow a
variable to hold multiple elements as a single value. These are great
for keeping multiple related values together. These collection types
are:

\begin{itemize}
\itemsep1pt\parskip0pt\parsep0pt
\item
  Tuples
\item
  Lists
\item
  Sets
\item
  Dictionaries
\end{itemize}

Now, let's start with one of the most basic data types, the integer.

    \section{Integers}\label{integers}

Integers are the discrete counting numbers that we've been using since
you started counting with your fingers. Even without creating any
variables we can still do basic arithmetic.

All of the basic arithmetic operations are the same as if we were to
write them out on paper.

    \begin{Verbatim}[commandchars=\\\{\}]
{\color{incolor}In [{\color{incolor}8}]:} \PY{l+m+mi}{2} \PY{o}{+} \PY{l+m+mi}{2}
\end{Verbatim}

            \begin{Verbatim}[commandchars=\\\{\}]
{\color{outcolor}Out[{\color{outcolor}8}]:} 4
\end{Verbatim}
        
    \begin{Verbatim}[commandchars=\\\{\}]
{\color{incolor}In [{\color{incolor}9}]:} \PY{l+m+mi}{4} \PY{o}{\PYZhy{}} \PY{l+m+mi}{2}
\end{Verbatim}

            \begin{Verbatim}[commandchars=\\\{\}]
{\color{outcolor}Out[{\color{outcolor}9}]:} 2
\end{Verbatim}
        
    \begin{Verbatim}[commandchars=\\\{\}]
{\color{incolor}In [{\color{incolor}10}]:} \PY{l+m+mi}{2} \PY{o}{*} \PY{l+m+mi}{2}
\end{Verbatim}

            \begin{Verbatim}[commandchars=\\\{\}]
{\color{outcolor}Out[{\color{outcolor}10}]:} 4
\end{Verbatim}
        
    We can also store the result of an operation into a variable. The
variable will store the evaluated answer, not the arithmetic expression.

    \begin{Verbatim}[commandchars=\\\{\}]
{\color{incolor}In [{\color{incolor}11}]:} \PY{n}{first\PYZus{}result} \PY{o}{=} \PY{l+m+mi}{8} \PY{o}{/} \PY{l+m+mi}{3}
         \PY{n}{first\PYZus{}result}
\end{Verbatim}

            \begin{Verbatim}[commandchars=\\\{\}]
{\color{outcolor}Out[{\color{outcolor}11}]:} 2.6666666666666665
\end{Verbatim}
        
    We see that the division operator stores the answer that we are used to,
which is \(2.6\bar{6}\). This behavior for the division operator is
actually new in Python 3! Before in Python 2 when we would do the
operation

\texttt{first\_result\ =\ 8\ /\ 3}

We would get the result:

\texttt{print(\ first\_result\ )\ ==\textgreater{}\ 2}

This was because it was thought that if we divide one integer by another
integer, that the operation should return an integer in order to keep
all the variable types the same.

To access this form of truncating division (it's called truncating
division, because it just truncates all the numbers after the decimal)
we actually use \texttt{//} like this:

    \begin{Verbatim}[commandchars=\\\{\}]
{\color{incolor}In [{\color{incolor}12}]:} \PY{n}{second\PYZus{}result} \PY{o}{=} \PY{l+m+mi}{8} \PY{o}{/}\PY{o}{/} \PY{l+m+mi}{3}
         \PY{n}{second\PYZus{}result}
\end{Verbatim}

            \begin{Verbatim}[commandchars=\\\{\}]
{\color{outcolor}Out[{\color{outcolor}12}]:} 2
\end{Verbatim}
        
    If we wanted to get the remainder, we would use \texttt{\%}

    \begin{Verbatim}[commandchars=\\\{\}]
{\color{incolor}In [{\color{incolor}13}]:} \PY{l+m+mi}{5} \PY{o}{\PYZpc{}} \PY{l+m+mi}{2}
\end{Verbatim}

            \begin{Verbatim}[commandchars=\\\{\}]
{\color{outcolor}Out[{\color{outcolor}13}]:} 1
\end{Verbatim}
        
    and it even works with a decimal.

    \begin{Verbatim}[commandchars=\\\{\}]
{\color{incolor}In [{\color{incolor}14}]:} \PY{l+m+mf}{4.2} \PY{o}{\PYZpc{}} \PY{l+m+mi}{2}
\end{Verbatim}

            \begin{Verbatim}[commandchars=\\\{\}]
{\color{outcolor}Out[{\color{outcolor}14}]:} 0.20000000000000018
\end{Verbatim}
        
    \section{Floats}\label{floats}

And that was a perfect introduction to a float, which is simply a number
with a decimal.

    \begin{Verbatim}[commandchars=\\\{\}]
{\color{incolor}In [{\color{incolor}15}]:} \PY{n}{new\PYZus{}float} \PY{o}{=} \PY{l+m+mf}{4.0}
         \PY{k}{print}\PY{p}{(}\PY{n}{new\PYZus{}float}\PY{p}{)}
\end{Verbatim}

    \begin{Verbatim}[commandchars=\\\{\}]
4.0
    \end{Verbatim}

    Wait, so what if I want to change an integer to a float or vice versa???

All we have to do is cast the number using the data type name that we
want to transform it to. We can see that below.

    \begin{Verbatim}[commandchars=\\\{\}]
{\color{incolor}In [{\color{incolor}16}]:} \PY{n+nb}{int}\PY{p}{(}\PY{l+m+mf}{4.8}\PY{p}{)}
\end{Verbatim}

            \begin{Verbatim}[commandchars=\\\{\}]
{\color{outcolor}Out[{\color{outcolor}16}]:} 4
\end{Verbatim}
        
    \begin{Verbatim}[commandchars=\\\{\}]
{\color{incolor}In [{\color{incolor}17}]:} \PY{n+nb}{float}\PY{p}{(}\PY{l+m+mi}{2}\PY{p}{)}
\end{Verbatim}

            \begin{Verbatim}[commandchars=\\\{\}]
{\color{outcolor}Out[{\color{outcolor}17}]:} 2.0
\end{Verbatim}
        
    If you're ever confused or interested about what the type of a variable
is you can always just check it with \texttt{type}

    \begin{Verbatim}[commandchars=\\\{\}]
{\color{incolor}In [{\color{incolor}18}]:} \PY{n+nb}{type}\PY{p}{(}\PY{n}{new\PYZus{}float}\PY{p}{)}
\end{Verbatim}

            \begin{Verbatim}[commandchars=\\\{\}]
{\color{outcolor}Out[{\color{outcolor}18}]:} float
\end{Verbatim}
        
    \begin{Verbatim}[commandchars=\\\{\}]
{\color{incolor}In [{\color{incolor}19}]:} \PY{n+nb}{type}\PY{p}{(}\PY{l+m+mi}{2}\PY{p}{)}
\end{Verbatim}

            \begin{Verbatim}[commandchars=\\\{\}]
{\color{outcolor}Out[{\color{outcolor}19}]:} int
\end{Verbatim}
        
    Now something that you should notice here is that \texttt{float} and
\texttt{int} are colored green (as is \texttt{print} and \texttt{type}).
That's because these are special words in Python that are already
defined by the language. However, Python will let you overwrite them but
it really is best to never do that! But you can see that if you ever
accidentally do it, like so

    \begin{Verbatim}[commandchars=\\\{\}]
{\color{incolor}In [{\color{incolor}20}]:} \PY{n+nb}{int} \PY{o}{=} \PY{l+m+mi}{4}
         \PY{k}{print}\PY{p}{(}\PY{l+s}{\PYZdq{}}\PY{l+s}{What have we don to int }\PY{l+s}{\PYZdq{}}\PY{p}{,} \PY{n+nb}{int}\PY{p}{)}
         \PY{n+nb}{int}\PY{p}{(}\PY{l+m+mf}{5.0}\PY{p}{)}
\end{Verbatim}

    \begin{Verbatim}[commandchars=\\\{\}]
What have we don to int  4
    \end{Verbatim}

    \begin{Verbatim}[commandchars=\\\{\}]

        ---------------------------------------------------------------------------
    TypeError                                 Traceback (most recent call last)

        <ipython-input-20-3be9f679579b> in <module>()
          1 int = 4
          2 print("What have we don to int ", int)
    ----> 3 int(5.0)
    

        TypeError: 'int' object is not callable

    \end{Verbatim}

    We'll lose the behavior of the function. However, we can get it back if
we just delete the assignment that we made.

To do that we use the \texttt{del} operator, which will delete a
variable and its assignment. It's another one of the special words in
Python.

    \begin{Verbatim}[commandchars=\\\{\}]
{\color{incolor}In [{\color{incolor}21}]:} \PY{k}{del} \PY{n+nb}{int}
         \PY{n+nb}{int}\PY{p}{(}\PY{l+m+mf}{5.0}\PY{p}{)}
\end{Verbatim}

            \begin{Verbatim}[commandchars=\\\{\}]
{\color{outcolor}Out[{\color{outcolor}21}]:} 5
\end{Verbatim}
        
    Moving forward with arithmetic, we can make entire mathematical
expressions. Just like when we first learned algebra, Python respects
the order of operations when it evaluates expressions (PEMDAS -
Parentheses, Exponents, Multiplication, Division, Addition,
Subtraction).

If we want to use an exponent we just use the \texttt{**} operator

    \begin{Verbatim}[commandchars=\\\{\}]
{\color{incolor}In [{\color{incolor}22}]:} \PY{l+m+mi}{2} \PY{o}{*}\PY{o}{*} \PY{l+m+mi}{3}
\end{Verbatim}

            \begin{Verbatim}[commandchars=\\\{\}]
{\color{outcolor}Out[{\color{outcolor}22}]:} 8
\end{Verbatim}
        
    \begin{Verbatim}[commandchars=\\\{\}]
{\color{incolor}In [{\color{incolor}23}]:} \PY{n}{eqn1} \PY{o}{=} \PY{l+m+mi}{2} \PY{o}{*} \PY{l+m+mi}{3} \PY{o}{\PYZhy{}} \PY{l+m+mi}{2}
         \PY{k}{print}\PY{p}{(} \PY{n}{eqn1} \PY{p}{)}
\end{Verbatim}

    \begin{Verbatim}[commandchars=\\\{\}]
4
    \end{Verbatim}

    \begin{Verbatim}[commandchars=\\\{\}]
{\color{incolor}In [{\color{incolor}24}]:} \PY{n}{eqn2} \PY{o}{=} \PY{o}{\PYZhy{}}\PY{l+m+mi}{2} \PY{o}{+} \PY{l+m+mi}{2} \PY{o}{*} \PY{l+m+mi}{3}
         \PY{k}{print}\PY{p}{(} \PY{n}{eqn2} \PY{p}{)}
\end{Verbatim}

    \begin{Verbatim}[commandchars=\\\{\}]
4
    \end{Verbatim}

    \begin{Verbatim}[commandchars=\\\{\}]
{\color{incolor}In [{\color{incolor}25}]:} \PY{n}{eqn3} \PY{o}{=} \PY{o}{\PYZhy{}}\PY{l+m+mi}{2} \PY{o}{+} \PY{p}{(}\PY{l+m+mi}{2} \PY{o}{\PYZpc{}} \PY{l+m+mi}{3}\PY{p}{)}
         \PY{k}{print}\PY{p}{(} \PY{n}{eqn3} \PY{p}{)}
\end{Verbatim}

    \begin{Verbatim}[commandchars=\\\{\}]
0
    \end{Verbatim}

    \begin{Verbatim}[commandchars=\\\{\}]
{\color{incolor}In [{\color{incolor}26}]:} \PY{n}{eqn4} \PY{o}{=} \PY{p}{(}\PY{o}{.}\PY{l+m+mi}{3} \PY{o}{+} \PY{l+m+mi}{5}\PY{p}{)} \PY{o}{/}\PY{o}{/} \PY{l+m+mi}{2}
         \PY{k}{print}\PY{p}{(}\PY{n}{eqn4}\PY{p}{)}
\end{Verbatim}

    \begin{Verbatim}[commandchars=\\\{\}]
2.0
    \end{Verbatim}

    Python also defines equivalencies: * \texttt{==} * \texttt{!=} *
\texttt{\textless{}=} * \texttt{\textgreater{}=} * \texttt{\textless{}}
* \texttt{\textgreater{}}

    The \(==\) operator let's us check if one side of the operator is equal
to the other side.

    \begin{Verbatim}[commandchars=\\\{\}]
{\color{incolor}In [{\color{incolor}1}]:} \PY{l+m+mi}{4} \PY{o}{==} \PY{l+m+mi}{4}
\end{Verbatim}

            \begin{Verbatim}[commandchars=\\\{\}]
{\color{outcolor}Out[{\color{outcolor}1}]:} True
\end{Verbatim}
        
    \begin{Verbatim}[commandchars=\\\{\}]
{\color{incolor}In [{\color{incolor}2}]:} \PY{l+m+mi}{4} \PY{o}{==} \PY{l+m+mi}{5}
\end{Verbatim}

            \begin{Verbatim}[commandchars=\\\{\}]
{\color{outcolor}Out[{\color{outcolor}2}]:} False
\end{Verbatim}
        
    We see here that Python evaluates the expression and if it is correct it
tells us that it is \texttt{True} or if it is incorrect it tells us that
it is \texttt{False}.

The \(!=\) operator let's us see if one side does not equal the other
side

    \begin{Verbatim}[commandchars=\\\{\}]
{\color{incolor}In [{\color{incolor}29}]:} \PY{l+m+mi}{4} \PY{o}{!=} \PY{l+m+mi}{2}
\end{Verbatim}

            \begin{Verbatim}[commandchars=\\\{\}]
{\color{outcolor}Out[{\color{outcolor}29}]:} True
\end{Verbatim}
        
    \begin{Verbatim}[commandchars=\\\{\}]
{\color{incolor}In [{\color{incolor}30}]:} \PY{l+m+mi}{4} \PY{o}{!=} \PY{l+m+mi}{4}
\end{Verbatim}

            \begin{Verbatim}[commandchars=\\\{\}]
{\color{outcolor}Out[{\color{outcolor}30}]:} False
\end{Verbatim}
        
    And then the greater than, less than, greater than or equal to, or less
than or equal to operators all work as we would expect.

    \begin{Verbatim}[commandchars=\\\{\}]
{\color{incolor}In [{\color{incolor}31}]:} \PY{l+m+mi}{4} \PY{o}{\PYZgt{}} \PY{l+m+mi}{2}
\end{Verbatim}

            \begin{Verbatim}[commandchars=\\\{\}]
{\color{outcolor}Out[{\color{outcolor}31}]:} True
\end{Verbatim}
        
    \begin{Verbatim}[commandchars=\\\{\}]
{\color{incolor}In [{\color{incolor}32}]:} \PY{l+m+mi}{4} \PY{o}{\PYZgt{}} \PY{l+m+mi}{4}
\end{Verbatim}

            \begin{Verbatim}[commandchars=\\\{\}]
{\color{outcolor}Out[{\color{outcolor}32}]:} False
\end{Verbatim}
        
    \begin{Verbatim}[commandchars=\\\{\}]
{\color{incolor}In [{\color{incolor}33}]:} \PY{l+m+mi}{4} \PY{o}{\PYZgt{}}\PY{o}{=} \PY{l+m+mi}{4}
\end{Verbatim}

            \begin{Verbatim}[commandchars=\\\{\}]
{\color{outcolor}Out[{\color{outcolor}33}]:} True
\end{Verbatim}
        
    \section{Booleans}\label{booleans}

Testing equivalencies is a perfect introduction to our next variable
type, the Boolean. In its most basic form, a Boolean is just
\texttt{True} or \texttt{False}.

With just these two variables we can implement basic logic and check for
truth in a programming language. Let's say that I have just one puppy at
home and his name is Frankenstein. So I will say that the variable puppy
is True.

    \begin{Verbatim}[commandchars=\\\{\}]
{\color{incolor}In [{\color{incolor}34}]:} \PY{n}{puppy} \PY{o}{=} \PY{n+nb+bp}{True}
\end{Verbatim}

    \begin{Verbatim}[commandchars=\\\{\}]
{\color{incolor}In [{\color{incolor}35}]:} \PY{k}{print}\PY{p}{(}\PY{n}{puppy}\PY{p}{)}
\end{Verbatim}

    \begin{Verbatim}[commandchars=\\\{\}]
True
    \end{Verbatim}

    \begin{Verbatim}[commandchars=\\\{\}]
{\color{incolor}In [{\color{incolor}36}]:} \PY{n+nb}{type}\PY{p}{(}\PY{n}{puppy}\PY{p}{)}
\end{Verbatim}

            \begin{Verbatim}[commandchars=\\\{\}]
{\color{outcolor}Out[{\color{outcolor}36}]:} bool
\end{Verbatim}
        
    Here we can see that when we print \texttt{puppy} it says \texttt{True}
and that the type is \texttt{bool}.

Since I only have one puppy, I'm going to say that \texttt{puppies} is
\texttt{False}.

    \begin{Verbatim}[commandchars=\\\{\}]
{\color{incolor}In [{\color{incolor}37}]:} \PY{n}{puppies} \PY{o}{=} \PY{n+nb+bp}{False}
\end{Verbatim}

    In Python we could have just created both of those variables at the same
time. I'll show that below, but the important thing is that the number
of variables on the left-hand side of the assignment operator
(\texttt{=}) should be equal to the number of variables on the
right-hand side.

    \begin{Verbatim}[commandchars=\\\{\}]
{\color{incolor}In [{\color{incolor}38}]:} \PY{n}{puppy}\PY{p}{,} \PY{n}{puppies} \PY{o}{=} \PY{n+nb+bp}{True}\PY{p}{,} \PY{n+nb+bp}{False}
\end{Verbatim}

    and we'll see here that each of those variables has its own value.

    \begin{Verbatim}[commandchars=\\\{\}]
{\color{incolor}In [{\color{incolor}39}]:} \PY{k}{print}\PY{p}{(}\PY{l+s}{\PYZdq{}}\PY{l+s}{Do I have a puppy?}\PY{l+s}{\PYZdq{}}\PY{p}{,} \PY{n}{puppy}\PY{p}{)}
         \PY{k}{print}\PY{p}{(}\PY{l+s}{\PYZdq{}}\PY{l+s}{Do I have puppies?}\PY{l+s}{\PYZdq{}}\PY{p}{,} \PY{n}{puppies}\PY{p}{)}
\end{Verbatim}

    \begin{Verbatim}[commandchars=\\\{\}]
Do I have a puppy? True
Do I have puppies? False
    \end{Verbatim}

    To implement logic statements we have three operations: \texttt{and},
\texttt{not}, and \texttt{or}.

If I use the \texttt{and} operator, then Python both sides of the
\texttt{and} expression need to be True for the expression to be true.

    \begin{Verbatim}[commandchars=\\\{\}]
{\color{incolor}In [{\color{incolor}40}]:} \PY{n+nb+bp}{True} \PY{o+ow}{and} \PY{n+nb+bp}{True}
\end{Verbatim}

            \begin{Verbatim}[commandchars=\\\{\}]
{\color{outcolor}Out[{\color{outcolor}40}]:} True
\end{Verbatim}
        
    If one side of the expression is \texttt{False}, then the whole
expression will be \texttt{False}

    \begin{Verbatim}[commandchars=\\\{\}]
{\color{incolor}In [{\color{incolor}41}]:} \PY{n+nb+bp}{True} \PY{o+ow}{and} \PY{n+nb+bp}{False}
\end{Verbatim}

            \begin{Verbatim}[commandchars=\\\{\}]
{\color{outcolor}Out[{\color{outcolor}41}]:} False
\end{Verbatim}
        
    and as you would expect we can perform these expressions with variables
(remember that I only have one puppy).

    \begin{Verbatim}[commandchars=\\\{\}]
{\color{incolor}In [{\color{incolor}42}]:} \PY{n}{puppy} \PY{o+ow}{and} \PY{n}{puppies}
\end{Verbatim}

            \begin{Verbatim}[commandchars=\\\{\}]
{\color{outcolor}Out[{\color{outcolor}42}]:} False
\end{Verbatim}
        
    The \texttt{not} operator expects that the following value should be
\texttt{False} for the expression to be true. If the following value is
\texttt{True} or exists then it will say that the expression is
\texttt{False}

    \begin{Verbatim}[commandchars=\\\{\}]
{\color{incolor}In [{\color{incolor}43}]:} \PY{o+ow}{not} \PY{n}{puppies}
\end{Verbatim}

            \begin{Verbatim}[commandchars=\\\{\}]
{\color{outcolor}Out[{\color{outcolor}43}]:} True
\end{Verbatim}
        
    \begin{Verbatim}[commandchars=\\\{\}]
{\color{incolor}In [{\color{incolor}44}]:} \PY{o+ow}{not} \PY{n}{puppy}
\end{Verbatim}

            \begin{Verbatim}[commandchars=\\\{\}]
{\color{outcolor}Out[{\color{outcolor}44}]:} False
\end{Verbatim}
        
    and we can combine that with the \texttt{and} operator to make our
entire previous statement about my pets \texttt{True}

    \begin{Verbatim}[commandchars=\\\{\}]
{\color{incolor}In [{\color{incolor}45}]:} \PY{n}{puppy} \PY{o+ow}{and} \PY{o+ow}{not} \PY{n}{puppies}
\end{Verbatim}

            \begin{Verbatim}[commandchars=\\\{\}]
{\color{outcolor}Out[{\color{outcolor}45}]:} True
\end{Verbatim}
        
    Finally, the \texttt{or} operator just requires that \textbf{at least
one} side of the expression is \texttt{True} for the expression to be
\texttt{True}

    \begin{Verbatim}[commandchars=\\\{\}]
{\color{incolor}In [{\color{incolor}46}]:} \PY{n}{puppy} \PY{o+ow}{or} \PY{n}{puppies}
\end{Verbatim}

            \begin{Verbatim}[commandchars=\\\{\}]
{\color{outcolor}Out[{\color{outcolor}46}]:} True
\end{Verbatim}
        
    But we still need at least one side to be \texttt{True}

    \begin{Verbatim}[commandchars=\\\{\}]
{\color{incolor}In [{\color{incolor}47}]:} \PY{n+nb+bp}{False} \PY{o+ow}{or} \PY{n+nb+bp}{False}
\end{Verbatim}

            \begin{Verbatim}[commandchars=\\\{\}]
{\color{outcolor}Out[{\color{outcolor}47}]:} False
\end{Verbatim}
        
    \section{Strings}\label{strings}

Finally, we have our last basic data type: Strings. Text is something
that is intuitive to us as humans but dealing with it programmatically
can become complicated (especially when there is a lot of it and we
don't know it's structure to begin with!).

To start off easily let's just make some variables.

    \begin{Verbatim}[commandchars=\\\{\}]
{\color{incolor}In [{\color{incolor}48}]:} \PY{n}{hello} \PY{o}{=} \PY{l+s}{\PYZsq{}}\PY{l+s}{hello}\PY{l+s}{\PYZsq{}}
         
         \PY{k}{print}\PY{p}{(} \PY{n}{hello} \PY{p}{)}
\end{Verbatim}

    \begin{Verbatim}[commandchars=\\\{\}]
hello
    \end{Verbatim}

    Now, it's important to remember that the variable name does not need to
be the same as its value.

    \begin{Verbatim}[commandchars=\\\{\}]
{\color{incolor}In [{\color{incolor}49}]:} \PY{n}{falafel} \PY{o}{=} \PY{l+s}{\PYZsq{}}\PY{l+s}{gyro}\PY{l+s}{\PYZsq{}}
         
         \PY{k}{print}\PY{p}{(} \PY{n}{falafel} \PY{p}{)}
\end{Verbatim}

    \begin{Verbatim}[commandchars=\\\{\}]
gyro
    \end{Verbatim}

    We can even use basic math operators to add strings together and make a
longer string

    \begin{Verbatim}[commandchars=\\\{\}]
{\color{incolor}In [{\color{incolor}50}]:} \PY{k}{print}\PY{p}{(}\PY{l+s}{\PYZdq{}}\PY{l+s}{gyros}\PY{l+s}{\PYZdq{}} \PY{o}{+} \PY{l+s}{\PYZdq{}}\PY{l+s}{ and }\PY{l+s}{\PYZdq{}} \PY{o}{+} \PY{l+s}{\PYZdq{}}\PY{l+s}{falafel}\PY{l+s}{\PYZdq{}}\PY{p}{)}
\end{Verbatim}

    \begin{Verbatim}[commandchars=\\\{\}]
gyros and falafel
    \end{Verbatim}

    We can even just multiply a string to make it longer. Can you say
\texttt{gyros} seven times fast?

    \begin{Verbatim}[commandchars=\\\{\}]
{\color{incolor}In [{\color{incolor}51}]:} \PY{l+s}{\PYZdq{}}\PY{l+s}{gyros}\PY{l+s}{\PYZdq{}} \PY{o}{*} \PY{l+m+mi}{7}
\end{Verbatim}

            \begin{Verbatim}[commandchars=\\\{\}]
{\color{outcolor}Out[{\color{outcolor}51}]:} 'gyrosgyrosgyrosgyrosgyrosgyrosgyros'
\end{Verbatim}
        
    Because Python can!

However, we can only use mathematical operations that make sense and are
clear what should occur (that means additive operators). We can't divide
or subtract though.

    \begin{Verbatim}[commandchars=\\\{\}]
{\color{incolor}In [{\color{incolor}52}]:} \PY{l+s}{\PYZdq{}}\PY{l+s}{gyros}\PY{l+s}{\PYZdq{}}\PY{o}{/}\PY{l+s}{\PYZdq{}}\PY{l+s}{falafel}\PY{l+s}{\PYZdq{}}
\end{Verbatim}

    \begin{Verbatim}[commandchars=\\\{\}]

        ---------------------------------------------------------------------------
    TypeError                                 Traceback (most recent call last)

        <ipython-input-52-ec2273e01fb3> in <module>()
    ----> 1 "gyros"/"falafel"
    

        TypeError: unsupported operand type(s) for /: 'str' and 'str'

    \end{Verbatim}

    \begin{Verbatim}[commandchars=\\\{\}]
{\color{incolor}In [{\color{incolor}53}]:} \PY{l+s}{\PYZdq{}}\PY{l+s}{gyros}\PY{l+s}{\PYZdq{}} \PY{o}{\PYZhy{}} \PY{l+s}{\PYZdq{}}\PY{l+s}{falafel}\PY{l+s}{\PYZdq{}}
\end{Verbatim}

    \begin{Verbatim}[commandchars=\\\{\}]

        ---------------------------------------------------------------------------
    TypeError                                 Traceback (most recent call last)

        <ipython-input-53-df4656ac4e0f> in <module>()
    ----> 1 "gyros" - "falafel"
    

        TypeError: unsupported operand type(s) for -: 'str' and 'str'

    \end{Verbatim}

    Now we can see more easily that we can add strings and variables that
have string values together to create a longer string and set that
longer string to a variable.

    \begin{Verbatim}[commandchars=\\\{\}]
{\color{incolor}In [{\color{incolor}54}]:} \PY{n}{order} \PY{o}{=} \PY{n}{hello} \PY{o}{+} \PY{l+s}{\PYZsq{}}\PY{l+s}{, I would like a }\PY{l+s}{\PYZsq{}} \PY{o}{+} \PY{n}{falafel}
         
         \PY{k}{print}\PY{p}{(}\PY{n}{order}\PY{p}{)}
\end{Verbatim}

    \begin{Verbatim}[commandchars=\\\{\}]
hello, I would like a gyro
    \end{Verbatim}

    Hmmm, well it is correct as a sentence but we forgot to capitalize
\texttt{hello}! This gets to back to the fact that it is a lot easier
for us to see and recognize strings after a lifetime of seeing them than
it is for the computer. Fortunately, string variables have some built-in
methods that we use on the variables to help with these situations.

    \begin{Verbatim}[commandchars=\\\{\}]
{\color{incolor}In [{\color{incolor}55}]:} \PY{n}{order}\PY{o}{.}\PY{n}{capitalize}\PY{p}{(}\PY{p}{)}
\end{Verbatim}

            \begin{Verbatim}[commandchars=\\\{\}]
{\color{outcolor}Out[{\color{outcolor}55}]:} 'Hello, i would like a gyro'
\end{Verbatim}
        
    BAM! We can just use the capitalize() method on the order variable and
we will get the capitalized \texttt{hello}. An important thing to note
is that while it is printing the string with \texttt{Hello}, it didn't
actually change the order variable.

    \begin{Verbatim}[commandchars=\\\{\}]
{\color{incolor}In [{\color{incolor}56}]:} \PY{n}{order}
\end{Verbatim}

            \begin{Verbatim}[commandchars=\\\{\}]
{\color{outcolor}Out[{\color{outcolor}56}]:} 'hello, I would like a gyro'
\end{Verbatim}
        
    If we wanted to change the original \texttt{order} variable to the
capitalized version, we would need to set \texttt{order} equal to
\texttt{order} when we use the \texttt{capitalize()} function.

    \begin{Verbatim}[commandchars=\\\{\}]
{\color{incolor}In [{\color{incolor}57}]:} \PY{n}{order} \PY{o}{=} \PY{n}{order}\PY{o}{.}\PY{n}{capitalize}\PY{p}{(}\PY{p}{)}
         
         \PY{n}{order}
\end{Verbatim}

            \begin{Verbatim}[commandchars=\\\{\}]
{\color{outcolor}Out[{\color{outcolor}57}]:} 'Hello, i would like a gyro'
\end{Verbatim}
        
    There are three other functions that perform actions like
\texttt{capitalize()}, and those are:

\begin{itemize}
\itemsep1pt\parskip0pt\parsep0pt
\item
  \texttt{lower()}, makes the entire string lowercase
\item
  \texttt{upper()}, makes the entire string uppercase
\item
  \texttt{title()}, capitalizes every word in a string
\end{itemize}

    \begin{Verbatim}[commandchars=\\\{\}]
{\color{incolor}In [{\color{incolor}58}]:} \PY{n}{order}\PY{o}{.}\PY{n}{title}\PY{p}{(}\PY{p}{)}
\end{Verbatim}

            \begin{Verbatim}[commandchars=\\\{\}]
{\color{outcolor}Out[{\color{outcolor}58}]:} 'Hello, I Would Like A Gyro'
\end{Verbatim}
        
    But you'll notice that I screwed up a little bit by setting
\texttt{order} to its capitalized version of itself (the grammar nazis
reading along have probably been going crazy all this time!). When we
capitalized the string, we lost the capitalized \texttt{i} which is
necessary since it's the pronoun! Python is pretty smart, but it also
does exactly what we told it to do and the \texttt{capitalize()}
function only capitalizes the first letter in a string.

The simplest thing to do would be to go back and recreate the order
variable.

    \begin{Verbatim}[commandchars=\\\{\}]
{\color{incolor}In [{\color{incolor}59}]:} \PY{n}{order} \PY{o}{=} \PY{n}{hello}\PY{o}{.}\PY{n}{capitalize}\PY{p}{(}\PY{p}{)} \PY{o}{+} \PY{l+s}{\PYZsq{}}\PY{l+s}{, I would like a }\PY{l+s}{\PYZsq{}} \PY{o}{+} \PY{n}{falafel}
         
         \PY{n}{order}
\end{Verbatim}

            \begin{Verbatim}[commandchars=\\\{\}]
{\color{outcolor}Out[{\color{outcolor}59}]:} 'Hello, I would like a gyro'
\end{Verbatim}
        
    We could do this programmatically though by using some the other
built-in functions.

One way would be to \texttt{strip} away the \texttt{Hello,} at the
start. Python has \texttt{strip} which strips away characters from the
right side and \texttt{lstrip} which strips away characters starting
from the left.

    \begin{Verbatim}[commandchars=\\\{\}]
{\color{incolor}In [{\color{incolor}60}]:} \PY{n}{order}\PY{o}{.}\PY{n}{lstrip}\PY{p}{(}\PY{l+s}{\PYZsq{}}\PY{l+s}{Helo,}\PY{l+s}{\PYZsq{}}\PY{p}{)}
\end{Verbatim}

            \begin{Verbatim}[commandchars=\\\{\}]
{\color{outcolor}Out[{\color{outcolor}60}]:} ' I would like a gyro'
\end{Verbatim}
        
    Notice that I didn't need to put in \texttt{l} twice. That's because I
just put in all of the individual characters I want stripped and Python
goes and removes \textbf{any and all} instances of those characters
until it encounters a character that I did not tell it to strip. We can
test that by adding an \texttt{I} the next character that we see, but
not a space which comes before it.

    \begin{Verbatim}[commandchars=\\\{\}]
{\color{incolor}In [{\color{incolor}61}]:} \PY{n}{order}\PY{o}{.}\PY{n}{lstrip}\PY{p}{(}\PY{l+s}{\PYZsq{}}\PY{l+s}{Helo,I}\PY{l+s}{\PYZsq{}}\PY{p}{)}
\end{Verbatim}

            \begin{Verbatim}[commandchars=\\\{\}]
{\color{outcolor}Out[{\color{outcolor}61}]:} ' I would like a gyro'
\end{Verbatim}
        
    Same result! This is a handy way of thinking, we just want to strip away
the parts we don't want until we get to what we do want.

We can also check the contents of a string using built-in methods. For
example, we can make sure that all of the characters are alphabetical.

    \begin{Verbatim}[commandchars=\\\{\}]
{\color{incolor}In [{\color{incolor}62}]:} \PY{n}{hello}\PY{o}{.}\PY{n}{isalpha}\PY{p}{(}\PY{p}{)}
\end{Verbatim}

            \begin{Verbatim}[commandchars=\\\{\}]
{\color{outcolor}Out[{\color{outcolor}62}]:} True
\end{Verbatim}
        
    This is handy because we can have numbers that are strings

    \begin{Verbatim}[commandchars=\\\{\}]
{\color{incolor}In [{\color{incolor}63}]:} \PY{l+s}{\PYZsq{}}\PY{l+s}{4}\PY{l+s}{\PYZsq{}}\PY{o}{.}\PY{n}{isnumeric}\PY{p}{(}\PY{p}{)}
\end{Verbatim}

            \begin{Verbatim}[commandchars=\\\{\}]
{\color{outcolor}Out[{\color{outcolor}63}]:} True
\end{Verbatim}
        
    This is a way to test the contents of the string without knowing what's
inside it. This is important because sometimes we will read in text that
has numbers, but we'll want those numbers to become an integer or float
so we can mathematically manipulate them.

We can convert a string of numbers into an integer just by casting it
with the \texttt{int()} function.

    \begin{Verbatim}[commandchars=\\\{\}]
{\color{incolor}In [{\color{incolor}64}]:} \PY{n}{real\PYZus{}number} \PY{o}{=} \PY{n+nb}{int}\PY{p}{(}\PY{l+s}{\PYZsq{}}\PY{l+s}{4}\PY{l+s}{\PYZsq{}}\PY{p}{)}
         
         \PY{k}{print}\PY{p}{(} \PY{n}{real\PYZus{}number} \PY{p}{)}
         \PY{k}{print}\PY{p}{(} \PY{n+nb}{type}\PY{p}{(}\PY{n}{real\PYZus{}number}\PY{p}{)} \PY{p}{)}
\end{Verbatim}

    \begin{Verbatim}[commandchars=\\\{\}]
4
<class 'int'>
    \end{Verbatim}

    We can do the same thing for floats also.

    \begin{Verbatim}[commandchars=\\\{\}]
{\color{incolor}In [{\color{incolor}65}]:} \PY{n+nb}{float}\PY{p}{(}\PY{l+s}{\PYZsq{}}\PY{l+s}{4.2}\PY{l+s}{\PYZsq{}}\PY{p}{)} \PY{o}{*} \PY{l+m+mi}{2}
\end{Verbatim}

            \begin{Verbatim}[commandchars=\\\{\}]
{\color{outcolor}Out[{\color{outcolor}65}]:} 8.4
\end{Verbatim}
        
    However, we cannot do that with anything that has alphabetical
characters.

    \begin{Verbatim}[commandchars=\\\{\}]
{\color{incolor}In [{\color{incolor}66}]:} \PY{n+nb}{float}\PY{p}{(}\PY{l+s}{\PYZsq{}}\PY{l+s}{I would like 4.5 gyros}\PY{l+s}{\PYZsq{}}\PY{p}{)}
\end{Verbatim}

    \begin{Verbatim}[commandchars=\\\{\}]

        ---------------------------------------------------------------------------
    ValueError                                Traceback (most recent call last)

        <ipython-input-66-b7a074fdfbca> in <module>()
    ----> 1 float('I would like 4.5 gyros')
    

        ValueError: could not convert string to float: 'I would like 4.5 gyros'

    \end{Verbatim}

    \section{The more complicated parts of
Strings}\label{the-more-complicated-parts-of-strings}

We already saw that we can strip out part of a string and that we can
add strings together. In addition, strings also behave like a collection
data type. That is, strings can be cut up and individual characterss can
be accessed.

This is something that is unique to strings among all the basic Python
data types. If we have a number we can't access a single part of it and
have it be the same value. We know intuitively that \texttt{40} is not
the same as just \texttt{4} so we always need the entire value.

When we access just a single element of the string, that is called
\textbf{indexing}. To index a single element we just add
\texttt{{[}\ {]}} after the variable name and tell it the numeric index
of the element we want to access.

    \begin{Verbatim}[commandchars=\\\{\}]
{\color{incolor}In [{\color{incolor}67}]:} \PY{n}{falafel}
\end{Verbatim}

            \begin{Verbatim}[commandchars=\\\{\}]
{\color{outcolor}Out[{\color{outcolor}67}]:} 'gyro'
\end{Verbatim}
        
    \begin{Verbatim}[commandchars=\\\{\}]
{\color{incolor}In [{\color{incolor}68}]:} \PY{n}{falafel}\PY{p}{[}\PY{l+m+mi}{1}\PY{p}{]}
\end{Verbatim}

            \begin{Verbatim}[commandchars=\\\{\}]
{\color{outcolor}Out[{\color{outcolor}68}]:} 'y'
\end{Verbatim}
        
    Huh? I said that I wanted the first element but Python returned
\texttt{y} which is the second character in the \texttt{falafel}
variable.

Why is that???

In Python, like in most other programming languages, all sequences are
actually \textbf{zero-indexed}. That means that the numerical index for
the first element is actually \texttt{0}

    \begin{Verbatim}[commandchars=\\\{\}]
{\color{incolor}In [{\color{incolor}69}]:} \PY{n}{falafel}\PY{p}{[}\PY{l+m+mi}{0}\PY{p}{]}
\end{Verbatim}

            \begin{Verbatim}[commandchars=\\\{\}]
{\color{outcolor}Out[{\color{outcolor}69}]:} 'g'
\end{Verbatim}
        
    The counting after that position is normal. So if want the letter
\texttt{r} that is the \textbf{third} letter in the word \texttt{gyro},
the index will be \textbf{\texttt{2}}

    \begin{Verbatim}[commandchars=\\\{\}]
{\color{incolor}In [{\color{incolor}70}]:} \PY{n}{falafel}\PY{p}{[}\PY{l+m+mi}{2}\PY{p}{]}
\end{Verbatim}

            \begin{Verbatim}[commandchars=\\\{\}]
{\color{outcolor}Out[{\color{outcolor}70}]:} 'r'
\end{Verbatim}
        
    We can also access elements starting from the end of the string too, we
just need to use a negative number. To get the very last letter we use
the index \textbf{\texttt{-1}}. The end starts from \texttt{-1} because
\texttt{0} always means the first entry and there is no such thing as
\texttt{-0}

    \begin{Verbatim}[commandchars=\\\{\}]
{\color{incolor}In [{\color{incolor}71}]:} \PY{n}{falafel}\PY{p}{[}\PY{o}{\PYZhy{}}\PY{l+m+mi}{1}\PY{p}{]}
\end{Verbatim}

            \begin{Verbatim}[commandchars=\\\{\}]
{\color{outcolor}Out[{\color{outcolor}71}]:} 'o'
\end{Verbatim}
        
    From the end, the counting works just the same as from the start

    \begin{Verbatim}[commandchars=\\\{\}]
{\color{incolor}In [{\color{incolor}72}]:} \PY{n}{falafel}\PY{p}{[}\PY{o}{\PYZhy{}}\PY{l+m+mi}{2}\PY{p}{]}
\end{Verbatim}

            \begin{Verbatim}[commandchars=\\\{\}]
{\color{outcolor}Out[{\color{outcolor}72}]:} 'r'
\end{Verbatim}
        
    As you can see here, there is always more than one way to skin a cat
with programming. No one way to solve the problem is more \emph{correct}
than any other way, it just comes down to what makes sense for your
problem, your code, and the way that you think about it.

Something to be aware of though, is that if you try to access an element
\textbf{it must exist}. That means that since \texttt{gyro} is four
letters long, I cannot give it an index that is greater than \texttt{3}

    \begin{Verbatim}[commandchars=\\\{\}]
{\color{incolor}In [{\color{incolor}73}]:} \PY{n}{falafel}\PY{p}{[}\PY{l+m+mi}{5}\PY{p}{]}
\end{Verbatim}

    \begin{Verbatim}[commandchars=\\\{\}]

        ---------------------------------------------------------------------------
    IndexError                                Traceback (most recent call last)

        <ipython-input-73-9c7825952384> in <module>()
    ----> 1 falafel[5]
    

        IndexError: string index out of range

    \end{Verbatim}

    That gives us an error! So always make sure that when you access an
element that the index is within the range of how long the string is.

    \section{Slicing a string}\label{slicing-a-string}

What if we wanted to get out more than one element from a string? We can
do that too, it's called \textbf{slicing}!

The syntax for slicing is deceptively simple, the full syntax is:

\texttt{variable{[}start\_index\ :\ stop\_index\ :\ step{]}}

You'll see that all of the inputs go within the \texttt{{[}{]}} and the
\texttt{:} separates each input.

The \texttt{start\_index} tells python what index we want to start
getting elements from.

The \texttt{stop\_index} tells python what index that we want elements
\textbf{up to but not including}

The \texttt{step} tells python how many steps to take between elements
within the range. This means that we don't need to take every element.
We could instead take \textbf{every other} element if we specified a
\texttt{step} of \texttt{2}.

So if we wanted to get the \texttt{gy} from \texttt{gyros} we would do

    \begin{Verbatim}[commandchars=\\\{\}]
{\color{incolor}In [{\color{incolor}74}]:} \PY{n}{falafel}
\end{Verbatim}

            \begin{Verbatim}[commandchars=\\\{\}]
{\color{outcolor}Out[{\color{outcolor}74}]:} 'gyro'
\end{Verbatim}
        
    \begin{Verbatim}[commandchars=\\\{\}]
{\color{incolor}In [{\color{incolor}75}]:} \PY{n}{falafel}\PY{p}{[}\PY{l+m+mi}{0} \PY{p}{:} \PY{l+m+mi}{2} \PY{p}{:} \PY{l+m+mi}{1}\PY{p}{]}
\end{Verbatim}

            \begin{Verbatim}[commandchars=\\\{\}]
{\color{outcolor}Out[{\color{outcolor}75}]:} 'gy'
\end{Verbatim}
        
    Remember that the index of \texttt{2} means the third element, so we
specified that we wanted every letter from the first index \textbf{up
until} the third index, and we want every letter.

We could get every other letter from the first four letters like so:

    \begin{Verbatim}[commandchars=\\\{\}]
{\color{incolor}In [{\color{incolor}76}]:} \PY{n}{falafel}\PY{p}{[}\PY{l+m+mi}{0} \PY{p}{:} \PY{l+m+mi}{4} \PY{p}{:} \PY{l+m+mi}{2}\PY{p}{]}
\end{Verbatim}

            \begin{Verbatim}[commandchars=\\\{\}]
{\color{outcolor}Out[{\color{outcolor}76}]:} 'gr'
\end{Verbatim}
        
    However, you'll rarely see someone specify all of those inputs when they
slice. If you don't give all of the inputs Python just assumes the
defaults. Those are:

\begin{itemize}
\itemsep1pt\parskip0pt\parsep0pt
\item
  \texttt{start\_index} is the first index \texttt{0}
\item
  \texttt{stop\_index} is the last index \texttt{-1} (notice that this
  will always be the last character no matter how long the string is)
\item
  \texttt{step} of \texttt{1}
\end{itemize}

    \begin{Verbatim}[commandchars=\\\{\}]
{\color{incolor}In [{\color{incolor}77}]:} \PY{n}{falafel}\PY{p}{[}\PY{l+m+mi}{0} \PY{p}{:} \PY{l+m+mi}{2}\PY{p}{]}
\end{Verbatim}

            \begin{Verbatim}[commandchars=\\\{\}]
{\color{outcolor}Out[{\color{outcolor}77}]:} 'gy'
\end{Verbatim}
        
    \begin{Verbatim}[commandchars=\\\{\}]
{\color{incolor}In [{\color{incolor}78}]:} \PY{n}{falafel}\PY{p}{[}\PY{p}{:} \PY{l+m+mi}{2}\PY{p}{]}
\end{Verbatim}

            \begin{Verbatim}[commandchars=\\\{\}]
{\color{outcolor}Out[{\color{outcolor}78}]:} 'gy'
\end{Verbatim}
        
    \begin{Verbatim}[commandchars=\\\{\}]
{\color{incolor}In [{\color{incolor}79}]:} \PY{n}{falafel}\PY{p}{[}\PY{p}{:} \PY{l+m+mi}{4} \PY{p}{:} \PY{l+m+mi}{2}\PY{p}{]}
\end{Verbatim}

            \begin{Verbatim}[commandchars=\\\{\}]
{\color{outcolor}Out[{\color{outcolor}79}]:} 'gr'
\end{Verbatim}
        
    And like you noticed from the default listings, we can mix and match
positive and negative indices like so

    \begin{Verbatim}[commandchars=\\\{\}]
{\color{incolor}In [{\color{incolor}80}]:} \PY{n}{falafel}\PY{p}{[} \PY{p}{:} \PY{o}{\PYZhy{}}\PY{l+m+mi}{1} \PY{p}{:} \PY{l+m+mi}{2}\PY{p}{]}
\end{Verbatim}

            \begin{Verbatim}[commandchars=\\\{\}]
{\color{outcolor}Out[{\color{outcolor}80}]:} 'gr'
\end{Verbatim}
        
    One difference from accessing a single element though, is that we can
give a \texttt{stop\_index} that is bigger than the length of the
string. Python will just give us all of the possible characters that
exist happily.

    \begin{Verbatim}[commandchars=\\\{\}]
{\color{incolor}In [{\color{incolor}81}]:} \PY{n}{falafel}\PY{p}{[}\PY{p}{:} \PY{l+m+mi}{10}\PY{p}{]}
\end{Verbatim}

            \begin{Verbatim}[commandchars=\\\{\}]
{\color{outcolor}Out[{\color{outcolor}81}]:} 'gyro'
\end{Verbatim}
        
    However, in most situations that is poor practice and you should just
not give a \texttt{stop\_index} so that it returns all of the
characters.

    \begin{Verbatim}[commandchars=\\\{\}]
{\color{incolor}In [{\color{incolor}82}]:} \PY{n}{falafel}\PY{p}{[}\PY{p}{:}\PY{p}{]}
\end{Verbatim}

            \begin{Verbatim}[commandchars=\\\{\}]
{\color{outcolor}Out[{\color{outcolor}82}]:} 'gyro'
\end{Verbatim}
        
    An important note though is that the \texttt{start\_index} always has to
come before the \texttt{stop\_index}, otherwise we will get an empty
string

    \begin{Verbatim}[commandchars=\\\{\}]
{\color{incolor}In [{\color{incolor}83}]:} \PY{n}{falafel}\PY{p}{[}\PY{l+m+mi}{3} \PY{p}{:} \PY{l+m+mi}{1}\PY{p}{]}
\end{Verbatim}

            \begin{Verbatim}[commandchars=\\\{\}]
{\color{outcolor}Out[{\color{outcolor}83}]:} ''
\end{Verbatim}
        
    That's because there is no valid sequence moving from left to right
while reading the string that exists with those limits.

If we want it to return the slice but reversed, we actually control that
with the \texttt{step} input and tell it that we want the reverse slice

    \begin{Verbatim}[commandchars=\\\{\}]
{\color{incolor}In [{\color{incolor}84}]:} \PY{n}{falafel}\PY{p}{[}\PY{l+m+mi}{3} \PY{p}{:} \PY{l+m+mi}{1} \PY{p}{:} \PY{o}{\PYZhy{}}\PY{l+m+mi}{1}\PY{p}{]}
\end{Verbatim}

            \begin{Verbatim}[commandchars=\\\{\}]
{\color{outcolor}Out[{\color{outcolor}84}]:} 'or'
\end{Verbatim}
        
    \section{Exercises}\label{exercises}

    Use five mathematical operators (\texttt{+\ -\ *\ /\ **}) to produce the
number \texttt{4}

    \begin{Verbatim}[commandchars=\\\{\}]
{\color{incolor}In [{\color{incolor}84}]:} 
\end{Verbatim}

    Convert the output of one of those expressions to a \texttt{float}

    \begin{Verbatim}[commandchars=\\\{\}]
{\color{incolor}In [{\color{incolor}84}]:} 
\end{Verbatim}

    I have a string called pet\_shop that has all of the different pet
varieties in a store.

    \begin{Verbatim}[commandchars=\\\{\}]
{\color{incolor}In [{\color{incolor}85}]:} \PY{n}{pet\PYZus{}shop} \PY{o}{=} \PY{l+s}{\PYZsq{}}\PY{l+s}{dog cat hedgehog fish bird}\PY{l+s}{\PYZsq{}}
\end{Verbatim}

    Capitalize all of the different pet types in a single line

    \begin{Verbatim}[commandchars=\\\{\}]
{\color{incolor}In [{\color{incolor}85}]:} 
\end{Verbatim}

    Print out a single \texttt{g} from \texttt{pet\_shop}

    \begin{Verbatim}[commandchars=\\\{\}]
{\color{incolor}In [{\color{incolor}85}]:} 
\end{Verbatim}

    Print out just \texttt{hedgehog}

    \begin{Verbatim}[commandchars=\\\{\}]
{\color{incolor}In [{\color{incolor}85}]:} 
\end{Verbatim}

    Print out \texttt{gohegdeh}

    \begin{Verbatim}[commandchars=\\\{\}]
{\color{incolor}In [{\color{incolor}85}]:} 
\end{Verbatim}

    I have two variables

    \begin{Verbatim}[commandchars=\\\{\}]
{\color{incolor}In [{\color{incolor}86}]:} \PY{n}{dogs}\PY{p}{,} \PY{n}{cats} \PY{o}{=} \PY{l+s}{\PYZsq{}}\PY{l+s}{8}\PY{l+s}{\PYZsq{}}\PY{p}{,} \PY{l+s}{\PYZsq{}}\PY{l+s}{4}\PY{l+s}{\PYZsq{}}
\end{Verbatim}

    that tell me how many \texttt{dogs} and \texttt{cat} I have at the
store. Using those two variables, calculate how many more dogs I have
than cats

    \begin{Verbatim}[commandchars=\\\{\}]
{\color{incolor}In [{\color{incolor}86}]:} 
\end{Verbatim}

    Exercises completed!

    \begin{Verbatim}[commandchars=\\\{\}]
{\color{incolor}In [{\color{incolor}87}]:} \PY{k+kn}{from} \PY{n+nn}{IPython.core.display} \PY{k+kn}{import} \PY{n}{HTML}
         
         
         \PY{k}{def} \PY{n+nf}{css\PYZus{}styling}\PY{p}{(}\PY{p}{)}\PY{p}{:}
             \PY{n}{styles} \PY{o}{=} \PY{n+nb}{open}\PY{p}{(}\PY{l+s}{\PYZdq{}}\PY{l+s}{../styles/custom.css}\PY{l+s}{\PYZdq{}}\PY{p}{,} \PY{l+s}{\PYZdq{}}\PY{l+s}{r}\PY{l+s}{\PYZdq{}}\PY{p}{)}\PY{o}{.}\PY{n}{read}\PY{p}{(}\PY{p}{)}
             \PY{k}{return} \PY{n}{HTML}\PY{p}{(}\PY{n}{styles}\PY{p}{)}
         \PY{n}{css\PYZus{}styling}\PY{p}{(}\PY{p}{)}
\end{Verbatim}

            \begin{Verbatim}[commandchars=\\\{\}]
{\color{outcolor}Out[{\color{outcolor}87}]:} <IPython.core.display.HTML at 0x1038db940>
\end{Verbatim}
        
    \begin{Verbatim}[commandchars=\\\{\}]
{\color{incolor}In [{\color{incolor}}]:} 
\end{Verbatim}


    % Add a bibliography block to the postdoc
    
    
    
    \end{document}
